\documentclass{scrartcl}

\title{Reducibility Among Combinatorial Problems \thanks{This research was partially supported by National Science Foundation Grant GJ-474}}
\author{Richard M. Karp \\ University of California at Berkeley}
\date{1972}

\begin{document}

\maketitle

\begin{abstract}
\emph{Abstract}: A large class of computational problems involve the determination of properties of graphs, digraphs, integers, arrays of integers, finite families of finite sets, boolean formulas and elements of other countable domains.
Through simple encodings from such domains into the set of words over a finite alphabet these problems can be converted into language recognition problems, and we can inquire into their computational complexity.
It is reasonable to consider such a problem satisfactorily solved when an algorithm for its solution is found which terminates within a number of steps bounded by a polynomial in the length of the input.
We show that a large number of classic unsolved problems of covering, matching, packing, routing, assignment and sequencing are equivalent, in the sense that either each of them possesses a polynomial-bounded algorithm or none of them does.
\end{abstract}

\section{Introduction}
All the general methods presently known for computing the chromatic number of a graph, deciding wheter a graph has a Hamilton circuit, or solving a system of linear inequalities in which the variables are constrained to be \(0\) or \(1\), require a combinatorial search for which the worst case time requirement grows exponentially with the length of the input.
In this paper we give theorems which strongly suggest, but do not imply, that these probems, as well as many others, will remain intractable perpetually.

We are specifically interested in the existence of algorithms that are guaranteed to terminate in a number of steps bounded by a polynomial in the length of the input.
We exhibit a class of well-known combinatorial problems, including those mentioned above, which are equivalent, in the sense that a polynomial-bounded algorithm for any one of them would effectively yield a polynomial-bounded algorithm for all.
We also show that, if these problems do possess polynomial-bounded algorithms then all the problems in an unexpectedly  wide class (roughly speaking, the class of problems solvable by polynomial-depth backtrack search) possess polynomial-bounded algorithms.

The following is a brief summary of the contents of the paper.
For the sake of definiteness our technical development is carried out in terms of the recognition of languages by one-tape Turing machines, but any of a wide variety of other abstract models of computation would yield the same theory.
Let \(\Sigma^{*}\) be the set of all finite strings of \(0\)'s and \(1\)'s.
\end{document}
