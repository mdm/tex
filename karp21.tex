\documentclass{scrartcl}

\title{Reducibility Among Combinatorial Problems \thanks{This research was partially supported by National Science Foundation Grant GJ-474}}
\author{Richard M. Karp \\ University of California at Berkeley}
\date{1972}

\begin{document}

\maketitle

\begin{abstract}
\emph{Abstract}: A large class of computational problems involve the determination of properties of graphs, digraphs, integers, arrays of
integers, finite families of finite sets, boolean formulas and elements of other countable domains. Through simple encodings from such
domains into the set of words over a finite alphabet these problems can be converted into language recognition problems, and we can inquire
into their computational complexity. It is reasonable to consider such a problem satisfactorily solved when an algorithm for its solution
is found which terminates within a number of steps bounded by a polynomial in the length of the input. We show that a large number of classic
unsolved problems of covering, matching, packing, routing, assignment and sequencing are equivalent, in the sense that either each of them
possesses a polynomial-bounded algorithm or none of them does.
\end{abstract}

\section{Introduction}
All the general methods presently known for computing the chromatic number of a graph, deciding wheter a graph has a Hamilton circuit, or
solving a system of linear inequalities in which the variables are constrained to be \(0\) or \(1\), require a combinatorial search for which
the worst case time requirement grows exponentially with the length of the input. In this paper we give theorems which strongly suggest, but
do not imply, that these probems, as well as many others, will remain intractable perpetually.

We are specifically interested in the existence of algorithms that are guaranteed to terminate in a number of steps bounded by a polynomial
in the length of the input. We exhibit a class of well-known combinatorial problems, including those mentioned above, which are equivalent,
in the sense that a polynomial-bounded algorithm for any one of them would effectively yield a polynomial-bounded algorithm for all. We also
show that, if these problems do possess polynomial-bounded algorithms then all the problems in an unexpectedly  wide class (roughly speaking,
the class of problems solvable by polynomial-depth backtrack search) possess polynomial-bounded algorithms.

The following is a brief summary of the contents of the paper. For the sake of definiteness our technical development is carried out in terms
of the recognition of languages by one-tape Turing machines, but any of a wide variety of other abstract models of computation would yield the
same theory. Let \(\Sigma^{*}\) be the set of all finite strings of \(0\)'s and \(1\)'s. A subset of \(\Sigma^{*}\) is called a \emph{language}.
Let \(\mathcal{P}\) be the class of languages recognizable in polynomial time by one-tape deterministic Turing machines, and let \(\mathcal{NP}\)
be the class of languages recognizable in polynomial time by one-tape nondeterministic Turing machines. Let \(\Pi\) be the class of functions from
\(\Sigma^{*}\) into \(\Sigma^{*}\) computable in polynomial time by one-tape Turing machines. Let \(L\) and \(M\) be languages. We say \(L \propto M\)
(\(L\) \emph{is reducible to} \(M\)) if there is a function \(f \in \Pi\) such that \(f(x) \in M \Leftrightarrow x \in L\). If \(M \in \mathcal{P}\)
and \(L \propto M\) then \(L \in \mathcal{P}\). We call \(L\) and \(M\) equivalent if \(L \propto M\) and \(M \propto L\). Call \(L\) \emph{(polynomial)
complete} if \(L \in \mathcal{NP}\) and every language in \(\mathcal{NP}\) is reducible to \(L\). Either all complete languages are in \(\mathcal{P}\),
or none of them are. The former alternative holds if and only if \(\mathcal{P} = \mathcal{NP}\)

The main contribution of this paper is the demonstration that a large number of classic difficult computational problems, arising in fields such
as mathematical programming, graph theory, combinatorics, computational logic and switching theory, are complete (and hence equivalent) when
expressed in a natural way as language recognition problems.

This paper was stimulated by the work of Stephen Cook (1971), and rests on an important theorem which appears in his paper.
The author also wishes to acknowledge the substantial contributions of Eugene Lawler and Robert Tarjan.

\section{The Class \(\mathcal{P}\)}

There is a large class of important computational problems which involve the determination of properties of graphs, digraphs, integers, finite families of finite
sets, boolean formulas and elements of other countable domains.
It is a reasonable working hypothesis, championed originally by Jack Edmonds (1965) in connection with problems in graph theory and integer programming, and by now
widely accepted, that such a problem can be regarded as tractable if and only if there is an algorithm for its solution whose running time is bounded by a polynomial
in the size of the input.
In this section we introduce and begin to investigate the class of probems solvable in polynomial time.

We begin by giving an extremely general definition of ``deterministic algorithm'', computing a function from a countable domain \(D\) into a countable range \(R\).

\end{document}
